\documentclass[12pt,halfline,a4paper]{ouparticle}
% * 
%
% ^.
\usepackage{amsmath,amssymb}
\usepackage{natbib}
\usepackage{graphicx,rotating}
\usepackage{dsfont}

\newcommand{\Est}[1]{\hat{#1}}
\newcommand{\beginsupplement}{%
        \setcounter{table}{0}
        \renewcommand{\thetable}{S\arabic{table}}%
        \setcounter{figure}{0}
        \renewcommand{\thefigure}{S\arabic{figure}}%
     }
     

\begin{document}

\title{PopSim: towards a standardized database of population genetic simulations}

\author{%
\name{The PopSim Consortium}
\address{}
\email{}}


\abstract{The explosion in population genomic data data demands ever increasingly complex modes of analysis. While
the field has met this challenge to date, we have not yet created field-wide standards by which to measure the performance
and accuracy of new tools. Here we describe a new resource, the PopSim database, that attempts to rectify this situation. This
resource contains XXX...}

\date{\today}

\keywords{Population genetics, Simulation, Inference}

\maketitle


\section*{Introduction}
While population genetics has always used statistical methods to make inferences from data,
the degree of sophistication of the questions, models, data, and computational approaches
used have all increased over the past decade. Currently there exist myriad computational methods
that can infer the histories of populations, the distribution of fitness effects,
recombination rates, mutational biases, the extent of positive selection from single genomes,
genotype data, or full genome sequence data (XX some generous citations XX).
While en masse these methods have increased our understanding of the
impacts of genetic and evolutionary processes, very little has been done to systematically
benchmark the quality of inferences gleaned from computational population genetics. Here we
describe the popSim consortium project, an effort aimed at creating systematic simulation
standards across a variety of common study organisms that will empower rigorous comparison
of methods to usher in the next era of genetics data analysis.

Many population genetic inference methods often make different assumptions as
well as use different summaries of the genetic variation data, even when
attempting to address the same biological question. For example, there are at
least X methods to infer population size changes. The MSMC method allows for the
population to continuously change in size over time. Other popular methods, like
Dadi, typically allow for a smaller number of size changes to occur at certain
points throughout the population's history. Further, the types of data that the
methods use also differs. MSMC uses 1-4 whole genome sequences and leverages the
spatial patterns of genetic variation along the sequence. Dadi instead uses the
site frequency spectrum (SFS), which summarizes the frequencies of SNPs in the
sample, ignoring any information about the correlation structure (linkage
disequilibrium) among SNPs.

The degree to which modeling assumptions and using different summaries of the
data can affect the ultimate results has been challenging to assess. Yet, there
have been several clear examples of different methods yielding fundamentally
different conclusions. For example, MSMC methods applied to human genomes have
suggested large ancient (>100 years ago) ancestral population sizes and
bottlenecks that have not been detected by SFS-based methods (see Beichman et
al. 2017). These distinct models differ in how they fit various summaries of the
genetic variation data, suggesting that modeling choices can greatly affect the
performance of the inference.

Presumably, specific methods will perform better under certain scenarios.
However, because these conditions are typically unknown, empirical researchers
lack principled guidelines for deciding which statistical method is best suited
for to accurately answer their particular question. The need for empirical
guidance will only increase as researchers continue to generate more genomic
variation data from non-model taxa and wish to make inferences regarding
demography and selection.

One way to benchmark different methods for statistical inference in population
genetics is to apply the methods to simulated datasets where the true model
parameters are known. Methods that perform well will accurately infer these
parameters and will also provide reasonable estimates of uncertainty of the
estimated parameters. Indeed simulations studies are a standard ingredient in
most papers proposing new methods in population genetics. However, these
simulation studies typically do not completely compare the performance of
different statistical methods. As such, they fall short of being able to provide
practical guidelines to empirical researchers. Standard simulations usually fall
short because they often focus on a part of the parameter space where the new
method is likely to perform well. For example, XXXX. Secondly, the degree to
which different methods are compared can be quite variable.

One reason why simulation studies do not compare the performance of many methods
is that these simulation studies are challenging to carry out. There are several
practical impediments to carrying out the ideal benchmarking simulation studies.
First, while there are many published population genetic models available, their
parameters are often not easily accessible, often presented in supplemental
tables. Implementing these parameters into usable commands for simulations can
be a time-consuming and error prone process. Further, often papers present many
different models, and different authors may choose to select distinct models,
making comparison difficult. While improvements in simulation software and ever
faster computers have partially emialoritated this concern, generating the
simulated datasets stil can still be a time and computing resource bottleneck.

For these reasons, we have generated a standard resource of simulated datasets
under a variety of population genetic models models to enable benchmarking of
new population genetic methods. We term this new resource, the PopSim project.
It includes…… Can be accessed at XXXX>

\section*{Results}
\paragraph{The PopSim Resource}
To address these problems, we developed the stdpopsim package, a community
maintained library of empirical genome data and population genetics simulation
models, providing easy access to realistic simulations.  The package organised
by species, initially consisting of humans, drosophila and arabadopsis. A
species definition consists of two key elements.  Firstly, the library defines
some basic information about its genome, with information about chromosome
lengths, average mutation rates and so on. We also provide access to detailed
empirical information such as recombination maps, which model empirically
observed heterogeneity along chromosomes. As such maps are often large and hard
to find, we provide a mechanism to automatically download (and cache locally) a
set of known maps. For example, in humans we support the HapMapII (XX cite XX) and
Decode (XX cite XX) genetic maps. The second key element of a species description
within stdpopsim is a set of carefully curated population genetic model
descriptions from the literature, which are the basis for accurate simulations
of specific historical scenarios.

\paragraph{Figure 1 about here}

Given the genome data and simulation model descriptions defined within the
library, it is then straightforward to run accurate, standardised simulations
across a range of organisms. Stdpopsim has a Python API and a user-friendly
command line interface, allowing users with minimal experience direct access to
state-of-the-art simulations. Simulations are output in the “tree sequence”
format (Kelleher, Etheridge, and McVean 2016; Kelleher et al. 2018, 2019), which
contains complete genealogical information about the simulated samples, is
extremely compact, and can be processed very efficiently using the tskit library
(Kelleher, Etheridge, and McVean 2016; Kelleher et al. 2018). Currently,
stdpopsim uses the  msprime coalescent simulator (Kelleher, Etheridge, and
McVean 2016) as the simulation engine. We plan to include other simulation
engines such as SLiM (Haller et al. 2019);(Haller and Messer 2019) to support
scenarios that cannot be modelled under the coalescent framework.

The stdpopsim command line interface, by default, outputs citation information
for the models, genetic maps and simulation engines used in any particular run.
We hope that this will encourage users to appropriately acknowledge the
resources used in published work, and authors
of simulation models to contribute to our ongoing community-driven development process.

\subsection*{Single Population demographic models}
We begin by simulating data under previously published demographic models for
humans, Drosophila, and Arabidopsis. These models consider the history of only a
single population at a time,and  were inferred using a variety of different
methods.

For humans, we implemented a simplified version of the Tennessen et al. (2012)
model with only the African population modeled (expansion from the ancestral
population and recent growth).

For Drosophila melanogaster we implemented the model inferred in Sheehan & Song
(2016) which covered three epochs: ancestral, bottleneck, and re-expansion, each
with constant within-epoch population size.

For Arabidopsis thaliana, we implemented the model from in Durvasula et al.
(2017) inferred using MSMC-8. This model includes a continuous change in
population size over time, rather than pre-sepceciified epochs of different
population sizes.

\subsection*{Performance of methods to infer demographic history for single
populations}
To evaluate and compare the performance of inference analyze methods for single
population size history, we use Snakemake (cite) to create a pipeline to run the
comparing inference methods. Concretely, we compared msmc, smc++, and
stairwayplot (cite all) on simulated genomes sampled from a single population,
in each of the stdpopsim models described above. The pipeline was generalized to
run R replicates with C chromosomes for N samples, for a total of R x C
simulations for each demographic model. After simulation, the pipeline preparess
input files for each of the respective inference methods by grouping all
chromosomes, for each sample, into a single file. This step results in three
input files, for each simulation replicate R, pertaining to each of the
respective inference methods and derived from the same simulated tree sequences.
Each of the inference programs are then run in parallel, followed by plotting of
Ne estimates from each program.

(Conclusion putting everything together for single population models--Kirk will
fill in after the other parts are here)


\subsection*{Multi Population demographic models}
Stdpopsim implements three human multi-population demographic models: Gutenkunst
et al. (2009), Tennesen et al. (2012) , and Browning et al. (2018). Gutenkunst
et al. modeled the out of Africa event as well as the subsequent divergence of
the European and Asian populations based on data from the Environmental Genome
Project (citation needed). Tennessen et al. fit a two population variant of the
Gutenkunst model which did not include Asian populations, where they focused on
modeling recent human population growth in European and African populations to
explain the presence of rare variants in exome sequencing data. Browning et al.
modeled an admixed American population jointly with its ancestral African,
European, and Asian populations based on data from the Hispanic Community Health
Study/Study of Latinos and the African-American and European-American
populations from Memphis and Pittsburgh (citation needed). Together these models
contain features found in real data (e.g. bottlenecks, population growth,
admixture) which are pertinent in the context of method development.

For Drosophila we implemented the model from Li & Stephan (2006) which included
an ancestral population in Africa, then a population expansion with a population
split and bottleneck into a European population.

\subsection*{Performance of methods to infer demographic history for multiple
populations}

Outlining for now since this is not entirely finalized yet Compared performance
of multi-population inference methods by running simulations under
multi-population models for humans and Drosophila and inferring simulated
parameters using dadi and fastsimcoal (and smc++???) For simplicity, conducted
inference in dadi and FSC using an IM model with constant population sizes and
bi-directional migration For humans, this meant that, although we simulated
under the three-population Gutenkunst model, samples were taken only from the
African and European populations Since inferred model did not match model used
for simulations, compared inferred parameters to parameters expected based on
mean coalescence time in simulated models (Fig 3) In general, used default
settings and did not attempt to optimize dadi or fsc runs Used Snakemake to
manage pipeline for two population analysis Results from two population pipeline
demonstrate that the different methods tested were generally able to recover the
simulated parameters Although we present comparison of results from several
competing methods, our aim at this stage is not to provide an exhaustive
evaluation of these methods Our hope is that future work with this pipeline will
enable more detailed exploration of the strengths and weaknesses of the numerous
multi-population inference methods that are available to the population genetics
community


\section*{Discussion}


\section*{Methods}
\subsection*{Generating the PopSim resource}

\subsection*{Model QC procedures}
Each demographic model was implemented independently by two people working from
the relevant primary sources. Disagreements between complete implementations
were resolved between the implementers in consultation with the authors of the
source papers when necessary. A comparison of the two implementations is then
added as a unit test.Technical details of the QC process can be found in the
GitHub repository in the developer documentation.

\subsection*{Workflow for analysis of simulated data}
Snakemake pipeline for 1 pop models
Snakemake pipeline for 2 pop models


\section*{Extra text}
[This section describes the stdpopsim Python library. What it’s for, how it’s used and what it can do]
[First para: running accurate simulations is hard]
Running realistic population genetic simulations is a complex task today. It is
straightforward to run simple simulations of population models such as the
coalescent or Wright-Fisher using (e.g.) msprime or SLiM, but the process of
instantiating these models to accurately reflect the properties of natural
populations that are important for a particular study is fraught with
difficulties. There are typically three important inputs that must be decided:
the demographic model, recombination and mutation rates. Demographic models for
a particular species are typically chosen by examining the literature for
published model specifications, which can consist of dozens of parameters. These
parameters must be transcribed from the original publication and translated into
the input format required for the simulator in question. This is a difficult and
error prone task, and mistakes are common. Recombination and mutation rates are
simpler to describe and often a single genome-wide estimate is appropriate.
However, such estimates are updated frequently and there is wide variation in
the values used for the simulations of the same species. When more fine-grained
simulations of the local chromosome structure is required, empirical
recombination and mutation rate maps are used. However, there is no standard
format for such maps, and finding the maps can be challenging.

\end{document}
